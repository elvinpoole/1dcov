\documentclass[12pt]{article}
 \usepackage[margin=1in]{geometry} 
\usepackage{amsmath,amsthm,amssymb,amsfonts}

\usepackage{hyperref}
 
\newcommand{\N}{\mathbb{N}}
\newcommand{\Z}{\mathbb{Z}}
\newcommand{\beq}{\begin{equation}}
\newcommand{\eeq}{\begin{equation}}

\setlength{\parindent}{0pt}
\setlength{\parskip}{1em}

\usepackage{color}

 
\begin{document}

 
\title{Notes on 1D plot covariances}
%\author{Author}
\maketitle

The number of objects in pixel $i$ is $N_i$.

In the absense of clustering, $N_i$ is just a poisson sampling so the the covariance matrix is


\begin{equation}
{\bf {\rm Cov}_{w=0}}(N_i,N_j) = \delta_{ij}{\bar N}
\end{equation}

where $\delta_{ij}$ is the Kronecker delta function, and ${\bar N}$ is the average number count per pixel. 

Since $w_{\rm true}(\theta)$ is just the covariance of the overdensity field. When we add clusteirng, the covariance of $N_i$ looks something like this

\begin{equation}
{\bf {\rm Cov}}(N_i,N_j) = \delta_{ij}{\bar N} + {\bar N}^2 w_{\rm true}(\theta_{ij})
\end{equation}

where $\theta_{ij}$ is the separation between pixel $i$ and $j$. I think we are assuming the noise on ${\bar N}$ itself is small in the second term for now. It is a little unclear to me right now what to do when $i=j$ as there will be some beyond poission noise coming from the sample variance of the clustering. Maybe this depends on the size of the pixel? I will come back to this later.....

For the 1d plots we are summing many pixels in an SP bin $k$. We will call this sum $N^{(SP)}$ and we will call the number of pixels in SP bin $k$, $N^{\rm pix}_{k}$.

\begin{equation}
N^{(SP)}_{k} = \sum_{i { \ \rm in \ } k} N_{i}
\end{equation}

In the absence of clustering, the covariance of this would be 

\begin{equation}
{\bf {\rm Cov}_{w=0}}(N^{SP}_k, N^{SP}_l) = \delta_{kl} {\bar N} N^{\rm pix}_{k}
\end{equation}

with clustering it is

\begin{equation}
{\bf {\rm Cov}}(N^{SP}_k, N^{SP}_l) =  {\bf {\rm Cov}}( \sum_{i { \ \rm in \ } k} N_{i}, \sum_{j { \ \rm in \ } l} N_{j})
\end{equation}
\begin{equation}
{\bf {\rm Cov}}(N^{SP}_k, N^{SP}_l) =  \sum_{i { \ \rm in \ } k} \sum_{j { \ \rm in \ } l} {\bf {\rm Cov}}(N_i, N_j) 
\end{equation}
\begin{equation}
{\bf {\rm Cov}}(N^{SP}_k, N^{SP}_l) =  \sum_{i { \ \rm in \ } k} \sum_{j { \ \rm in \ } l} \left[ \delta_{ij}{\bar N} + {\bar N}^2 w_{\rm true}(\theta_{ij}) \right]
\end{equation}
\begin{equation}
{\bf {\rm Cov}}(N^{SP}_k, N^{SP}_l) =  \delta_{kl} {\bar N} N^{\rm pix}_{k} + \sum_{\theta} N^{\rm (pix \ pairs)}_{kl}(\theta) {\bar N}^2 w_{\rm true}(\theta) + {(i=j \ \rm term?)}
\end{equation}

where $N_{\rm pix \ pairs}(\theta)$ is the number of pairs of pixels separated by $\theta$ which can be obtained from treecorr. Assuming your $w(\theta)$ and pair counts are in discrete bins.

If there is an actual systematic signal, I think we could just use the real number counts as the poisson term

\begin{equation}
{\bf {\rm Cov}}(N^{SP}_k, N^{SP}_l) =  \delta_{kl} N^{SP}_{k} + \sum_{\theta} N^{\rm (pix \ pairs)}_{kl}(\theta) {\bar N}^2 w_{\rm true}(\theta) + {(i=j \ \rm term?)}
\end{equation}



\end{document}




